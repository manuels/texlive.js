%%%%%%%%%%%%%%%%%%%%%%%%%%%%%%%%%%%%%%%%%%%%%%%%%%%%%%%%%%%%%%%%%%%%%
%     CAP  ---  Macros for typesetting programs in C and Pascal     %
%       Micha\l{} Gulczy\'nski, Szczecin, Feb 1997 / Feb 1998       %
%                     mgulcz@we.tuniv.szczecin.pl                   %
%%%%%%%%%%%%%%%%%%%%%%%%%%%%%%%%%%%%%%%%%%%%%%%%%%%%%%%%%%%%%%%%%%%%%
%
%  version 1.2
%    This file contains macros for typesetting programs in Pascal.
%    It belongs to the CAP package. Do not distribute separately.
%
%    All font definitions are located in the cap_comm.tex file.
%

%
%      set of keywords; they must be separated with spaces
%      a space must precede the first one and follow the last one, too
%
\def\KeywordsPascal{
and as asm array begin case class const constructor destructor
div do downto else end except exports file finally for function goto
if implementation in inherited inline initialization interface is label
library mod nil not object of or packed procedure program property
raise record repeat set shl shr string then to try type unit until
uses var while with xor
}
\catcode`\@ = 11
%
%      that's where we begin
%
\def\BeginPascal{%
  \@PrepareCPas
  \@LetsStartPascal
}%
% Some people in Poland use so-called ``slash notation''
% to represent certain Polish letters --- in this situation
% slash is an active character. On the other hand we use slash
% in pathnames: directory/subdirectory/file. I made this part
% sooo complicated, bacause I had to neutralise slash in
% \InputPascal.
\def\@InputPascal#1{%
  \message{(Pascal: #1}%
  \openin\@InFile = #1
  \@PrepareCPas
  % The file is read line by line and each line is typeset
  % just like a separate program. Therefore the size of program
  % typeset using this macro is (almost) unlimited.
  \loop
    \global\read\@InFile to \@TextOfProgram
    \@WriteTextOfProgramPascal
  \if \neof\@InFile \repeat
  \closein\@InFile
  \endgroup    % this group was begun by \@PrepareCPas
  \endgroup    % this group was begun by \InputPascal
  \message{)}%
}
\def\InputPascal{%
  \begingroup
  \catcode`\/ = 11
  \@InputPascal
}%
%
%      delimiter \EndC will be ordinary text
%
{ \catcode`\|=0   \catcode`\\=12
  |gdef|@LetsStartPascal#1\EndPascal{%
    |gdef|@TextOfProgram{#1}%
    |@WriteTextOfProgramPascal
    |endgroup    % this group was begun by \@PrepareCPas
  }
}
%
%      macro \@TextOfProgram contains the whole text of program
%
\def\@WriteTextOfProgramPascal{%
  \expandafter\@ReadCharPascal\@TextOfProgram\@EndOfProgram
}
%
%      heart of the program -- the argument is a single char
%
\def\@ReadCharPascal#1{%
  % this macro calls itself until the argument #1 is \@EndOfProgram
  \if\@Identical{\string#1}{\string\@EndOfProgram}%
    \let\@WhatNext = \relax
  \else
    \let\@WhatNext = \@ReadCharPascal
    \global\@CharCode = `#1\relax
    \ifcase \@WhereAmI
      %%%%%%%%%%%%%%%%%%%
      % \@NothingSpecial:
      %%%%%%%%%%%%%%%%%%%
      \ifnum \@PrevChar=`\(
        \ifnum \@CharCode=`\*
          \global\@WhereAmI = \@LongComment
          \@CommentState
        \fi
        \@Output{(}%
      \fi
      % the longest possible string containing only letters and digits
      % is either a keyword or an identifier
      \if\@DigitLetter{\@CharCode}%
        \edef\@Word{\@Word#1}%
      \else
        \if\@Identical{\@Word}{}%
        \else
          \@WriteWord{\@Word}{\KeywordsPascal}%
          \def\@Word{}%
        \fi
        \ifnum \@CharCode=`\{
          \global\@WhereAmI = \@ShortComment
          \@CommentState
        \fi
        \ifnum \@CharCode=`\'
          \global\@WhereAmI = \@Text
          \@TextState
        \fi
        \ifnum \@CharCode=`\(
        \else
          \@WriteChar{#1}%
        \fi
      \fi
    \or
      %%%%%%%%%
      % \@Text:
      %%%%%%%%%
      \@WriteChar{#1}%
      \ifnum \@CharCode=`\'
        \global\@WhereAmI = \@NothingSpecial
        \@SymbolState
      \fi
    \or
      %%%%%%%%%%%%%%
      % \@Directive:
      %%%%%%%%%%%%%%
      % there are no directives in Pascal, but i'll leave it
    \or
      %%%%%%%%%%%%%%%%%
      % \@ShortComment:
      %%%%%%%%%%%%%%%%%
      \@WriteChar{#1}%
      \ifnum \@CharCode=`\}
        \global\@WhereAmI = \@NothingSpecial
        \@SymbolState
      \fi
    \or
      %%%%%%%%%%%%%%%%
      % \@LongComment:
      %%%%%%%%%%%%%%%%
      \@WriteChar{#1}%
      \ifnum \@PrevChar=`\*
        \ifnum \@CharCode=`\)
          \global\@WhereAmI = \@NothingSpecial
          \@SymbolState
          \@CharCode = 32
        \fi
      \fi
    \fi
    \global\@PrevChar = \@CharCode
  \fi
  \@WhatNext
}

\ifx \@PrepareCPas \@Dont@Know@What@It@Is
  \input cap_comm
\fi

\catcode`\@ = 12
