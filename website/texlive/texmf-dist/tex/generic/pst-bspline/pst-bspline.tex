%% BEGIN pst-bspline.tex
%% Author: Michael Sharpe (msharpe at ucsd.edu)
%%
%\message{ v\fileversion, \filedate}
\csname PSTBsplineLoaded\endcsname
\let\PSTBsplineLoaded\endinput

\ifx\PSTricksLoaded\endinput  \else\input pstricks \fi
\ifx\PSTnodeLoaded\endinput   \else\input pst-node \fi
\ifx\PSTXKeyLoaded\endinput   \else\input pst-xkey \fi
\def\fileversion{1.61}
\def\filedate{2012/03/06}

\message{`pst-bspline' v\fileversion, \filedate\space Bspline routines for pstricks (ms)}
%
\edef\TheAtCode{\the\catcode`\@} \catcode`\@=11\relax
\pst@addfams{pst-bspline}
\pstheader{pst-bspline.pro}

\SpecialCoor

%\newcount\pst@args%used in several macros--now defined in pst-node.tex
\def\PST@root{}
\newcount\bsp@args
\newcount\bsp@cntA
\newcount\bsp@cntB
\newif\ifbsp@omitends
\newif\ifbsp@closed
%\newif\ifbsp@showpoints
%\newif\ifbsp@doubleline
\newif\ifbsp@seq
\newif\ifbsp@per %flag for periodic interpolation
%
\def\bsp@init{%these items are used only within the plotting macros
\def\bsp@arrowA{}%
\def\bsp@arrowB{}%
}%
\def\refreshbspopts{%
\bsp@closedfalse%
\bsp@omitendsfalse%
\bsp@seqfalse%
}%
% showframe key and its default are global
\define@boolkey[psset]{pst-bspline}[Pst@]{showframe}[true]{}
\psset[pst-bspline]{showframe=false}%
\define@boolkey[psset]{pst-bspline}[Pst@]{reverseclip}[true]{}
\psset[pst-bspline]{reverseclip=false}%
%
%Bspline drawing macros
\def\psBsplineNodes{\def\pst@par{}\pst@object{psBsplineNodes}}%
\def\psBsplineNodesC{\def\pst@par{}\pst@object{psBsplineCNodes}}%
\def\psBsplineNodesE{\def\pst@par{}\pst@object{psBsplineENodes}}%
\def\psBsplineC{\def\pst@par{}\pst@object{psBsplineC}}%
\def\psBsplineE{\def\pst@par{}\pst@object{psBsplineE}}%
\def\psBspline{\def\pst@par{}\pst@object{psBspline}}%
%Bspline internal macros, level 1
\def\psBsplineC@i{\bsp@init\refreshbspopts\bsp@closedtrue\ps@bspline}%
\def\psBsplineE@i{\bsp@init\refreshbspopts\bsp@omitendstrue\ps@bspline}%
\def\psBspline@i{\bsp@init\refreshbspopts\ps@bspline}%
\def\psBsplineNodes@i{\bsp@init\refreshbspopts\bsp@seqtrue\ps@bsplineNodes}%
\def\psBsplineCNodes@i{\bsp@init\refreshbspopts\bsp@seqtrue\bsp@closedtrue%
\ps@bsplineNodes}%
\def\psBsplineENodes@i{\bsp@init\refreshbspopts\bsp@seqtrue\bsp@omitendstrue%
\ps@bsplineNodes}%
%Bspline internal macros, level 2
\def\ps@bsplineNodes#1#2{%noderoot, endindex
\def\bsp@root{#1}%
\bsp@args=#2\relax%
\next@Point%
}%
%
\def\ps@bspline{\bsp@args=0\relax%
\@ifnextchar({\psBspline@read{pn@de}}{\psBspline@read}}%
%   
\def\psBspline@read#1{%
\def\bsp@root{#1}%
\next@Point%
}%
%
\def\next@Point{\@ifnextchar({\next@Point@i}{\psBsplineMain}}%
\def\next@Point@i(#1){%
\pnode(#1){\bsp@root\the\bsp@args}%
\advance\bsp@args by \@ne%
\next@Point}%
% The main drawing routine
\def\psBsplineMain{%done reading nodes
\pst@killglue%
\begingroup%
\use@par%
\ifbsp@closed\begin@ClosedObj \else \begin@OpenObj \fi%
\let\bsp@linecolor\pslinecolor%
\let\bsp@arrowA\psk@arrowA%
\let\bsp@arrowB\psk@arrowB%
\let\bsp@linestyle\pslinestyle%
\ifbsp@seq\advance\bsp@args by \@ne\fi%
\ifbsp@closed%
  \psset{doubleline=false}%
  \pnode(\bsp@root0){\bsp@root\the\bsp@args}%
  \advance\bsp@args by \@ne%
  \pnode(\bsp@root1){\bsp@root\the\bsp@args}%
  \advance\bsp@args by \@ne%
\fi%
\advance\bsp@args by \m@ne%
% Define the left and right bezier control points in each interval. These are 
% denoted (root+) R0, L1, R1, L2, R2, etc
\multido{\ia=0+1,\ib=1+1}{\bsp@args}{%
\ifPst@showframe%
  \ncline[linestyle=dashed,linecolor=gray,arrows=*-*]{\bsp@root\ia}{\bsp@root\ib}%
\fi%
  \nodexn{.667(\bsp@root\ia)+.333(\bsp@root\ib)}{\bsp@root R\ia}%
  \nodexn{.333(\bsp@root\ia)+.667(\bsp@root\ib)}{\bsp@root L\ib}%
  }%
%Finally, define the bezier endpoints for each interval
\advance\bsp@args by \m@ne%
\mmultido{\ia=0+1}{\bsp@args}{%
\ifPst@showframe%
  \ncline[linestyle=solid,linecolor=red]{\bsp@root L\ia}{\bsp@root R\ia}%
\fi%
\midAB(\bsp@root L\ia)(\bsp@root R\ia){\bsp@root S\ia}%
\ifPst@showframe%
  \psdot[linecolor=red](\bsp@root S\ia)%
\fi%
}%
\ifbsp@closed%
  \pnode(\bsp@root S\the\bsp@args){\bsp@root S0}%
 \else%
   \advance\bsp@args by \@ne%
   \pnode(\bsp@root0){\bsp@root S0}%
   \pnode(\bsp@root\the\bsp@args){\bsp@root S\the\bsp@args}%
 \fi%
\advance\bsp@args by \m@ne%
%the pieces of bezier curve
\ifbsp@closed%
\pscustom[arrows=-]{%
\psbezier[showpoints=false](\bsp@root S0)(\bsp@root R0)(\bsp@root L1)(\bsp@root S1)%
\multido{\ia=1+1,\ib=2+1}{\bsp@args}{\psbezier(\bsp@root R\ia)(\bsp@root L\ib)%
(\bsp@root S\ib)}%
%within \pscustom, \psbezier takes 3 args, using last point of previous
}%
\else %open curve---make arrows, don't use \pscustom
 \advance\bsp@args by \m@ne%
 \ifbsp@omitends \bsp@cntA=2\relax \bsp@cntB=\@ne \advance\bsp@args by -2\relax%
 \else \bsp@cntA=\@ne\bsp@cntB=\z@ \fi%
 \psbezier[arrows=-\bsp@arrowA,showpoints=false](\bsp@root S\the\bsp@cntA)%
 (\bsp@root L\the\bsp@cntA)(\bsp@root R\the\bsp@cntB)(\bsp@root S\the\bsp@cntB)%
 \mmultido{\ia=\the\bsp@cntA+1,\ib=\the\bsp@cntB+1}{\bsp@args}%
 {\psbezier[arrows=-,showpoints=false](\bsp@root S\ib)(\bsp@root R\ib)%
 (\bsp@root L\ia)(\bsp@root S\ia)}%
 %
 \ifbsp@omitends \advance\bsp@args by \@ne \fi%
 \advance\bsp@args by \@ne%
 \bsp@cntA=\bsp@args%
 \bsp@cntB=\bsp@cntA%
 \advance\bsp@cntB by \@ne%
 \psbezier[arrows=-\bsp@arrowB,showpoints=false](\bsp@root S\the\bsp@cntA)%
 (\bsp@root R\the\bsp@cntA)(\bsp@root L\the\bsp@cntB)(\bsp@root S\the\bsp@cntB)%
% \ifPst@showframe \psdot[linecolor=red](\bsp@root L\the\bsp@cntB) \fi%
\fi% end \ifbsp@closed
%\ifPst@showframe% 
%  \psdot[linecolor=red](\bsp@root R0)%
%  \psdot[linecolor=red](\bsp@root L\the\bsp@args)%
%\fi%
\ifshowpoints%
  \multido{\i=0+1}{\bsp@args}{\psdot(\bsp@root S\i)}%
\fi%
 \ifbsp@closed%
   \end@ClosedObj%
 \else%
   \end@OpenObj%
 \fi%
\endgroup%
\ignorespaces%
}
%
\def\psbspline{\def\pst@par{}\pst@object{psbspline}}%
\def\psbspline@i{\pst@getarrows\psbspline@ii}%
\def\psbspline@ii{%
\pnodesX%make the following sequences into nodes named S@0,...
%last index=\pst@args
}%
\def\psbspline@iii{%
\pst@killglue%
\psBsplineInterp{S@}{\the\pst@args}%creates S@B0...
\begingroup%
\use@par%
\psBsplineNodes{S@B}{\the\pst@args}%draw curve
\endgroup\ignorespaces}%
%
\def\pnodesX{%version of \pnodes where the last step moves to \psbspline@iii
\pst@args=\z@%
\def\PST@root{S@}%
\next@NodeX}%
%
\def\next@NodeX{\@ifnextchar({\next@NodeX@i}{\advance\pst@args by \m@ne\typeout{Defined nodes \PST@root0 ..  \PST@root\the\pst@args.}\psbspline@iii}}%
%
\def\next@NodeX@i(#1){%
\pnode(#1){\PST@root\the\pst@args}%
\advance\pst@args\@ne%
\next@NodeX}%
%
\def\make@mcoeff#1{%
\pstVerb{ tx@Dict begin /mcoeff #1\space 1 add array def %
 mcoeff 1 0.25 put %
 1 1 #1\space 1 sub %
 {/j exch def %j is counter
 mcoeff j 1 add 1 4 mcoeff j get sub div put  } for  end }}%
 %
\def\psBsplineInterp#1#2{{%
% #1=node name root to interpolate, #2=end index, 
% eg, if we want to interpolate S0,..., S100,  #1=S, #2=100,
\pst@killglue
\global\bsp@perfalse %
\newcount\top@mone \top@mone=#2\relax%100
\newcount\top@ndx \top@ndx=\top@mone% 
\advance\top@ndx by -1\relax%99
\advance\top@mone by -2\relax%98
\newcount\top@mtwo \top@mtwo=\top@mone \advance\top@mtwo by \m@ne%97
\make@mcoeff{#2}%
\def\noderoot{#1}%
\pnode(\noderoot 0){\noderoot B0}%
\pnode(\noderoot #2){\noderoot B#2}%
\ifnum\top@mone>\m@ne% at least 3 points to interpolate
  \ifnum\top@mone>\z@% we started with 4 or more points to interpolate
	\psLCNode(\noderoot 1){6}(\noderoot 0){-1}{\noderoot B1}% 6*S1-S0->SB1
	\multido{\iA=2+1,\iAA=1+1}{\top@mone}{\psLCNodeVar(\noderoot\iA)(\noderoot B\iAA)(! 6 mcoeff \iAA\space get neg ){\noderoot B\iA}}%2 to n-2, 6*S_k-m_{k-1}*SB_{k-1}->SB_k
	\psLCNodeVar(\noderoot \the\top@ndx)(\noderoot #2)(! mcoeff \the\top@ndx\space get dup 6 mul exch neg  ){tmp@node}%m_{99}*(6*S99-S100)->tmp@node
	\psLCNodeVar(tmp@node)(\noderoot B\the\top@mone)(! 1 mcoeff \the\top@ndx\space get mcoeff \the\top@mone\space get mul neg ){\noderoot B\the\top@ndx}% temp@node -m_{98}*m_{99}*SB98->SB99
	\multido{\iA=\the\top@mone+-1,\iAA=\the\top@ndx+-1}{\the\top@mone}{\psLCNodeVar(\noderoot B\iA)(\noderoot B\iAA)(! mcoeff \iA\space get dup neg ){\noderoot B\iA}% m_k*SB_k-m_k*SB_{k+1}->SB_k, k=98 downto 1
	}% end multido
  \else% exactly 3 points to interpolate
    \nodexn{1.5(\noderoot1)-.25(\noderoot0)-.25(\noderoot2)}{\noderoot B1}%
\fi\fi%
}\ignorespaces}% end def
%
\def\psBsplineInterpC#1#2{{%
 % #1=node name root, #2=top index,
 % eg, if we want to interpolate S0,..., S100, #1=S, #2=100
 % interpolation in this case is periodic (closed curve)
\pst@killglue
\global\bsp@pertrue %
\pst@cnta=\csname#1nodecount\endcsname\advance\pst@cnta \@ne %
\newcount\top@mone \top@mone=#2\relax%
\ifnum \top@mone<\pst@cnta %
  \expandafter\xdef \csname #1nodecount\endcsname {\the\pst@cnta}%
  \typeout{Changed #1nodecount to \the\pst@cnta}%
  %\top@mone=\pst@cnta %
\else%
  \ifnum \top@mone>\pst@cnta \top@mone=\pst@cnta\advance\top@mone \m@ne\fi 
\fi%
\newcount\top@ndx \top@ndx=\top@mone% 
\advance\top@ndx by \@ne%number of nodes to interpolate
%\
\newcount\top@mtwo \top@mtwo=\top@mone \advance\top@mtwo\m@ne%
% After copying S0 to S101, \top@ndx, \top@mone, \top@mtwo are top index, top index-1, top index-two---101, 100, 99 in the example
\make@mcoeff{\the\top@ndx}%
\def\noderoot{#1}%
\pstVerb{ tx@Dict begin /bcoeff #2\space 2 add array def  /ccoeff #2\space 2 add array def /xcoeff #2\space 2 add array def /ycoeff #2\space 2 add array def %
%% initialize b, c, x ,y values
 1 1 \the\top@ndx\space %1 add
 {/j exch def %j is counter
 ccoeff j 0 put bcoeff j 0 put } for %
 bcoeff 1 1 put bcoeff #2\space 1 put bcoeff #2\space 1 add  4 put %
 ccoeff 1 0.25 put ccoeff #2\space 1 put end }%
%end \pstVerb
\pnode(\noderoot 0){\noderoot\the\top@ndx}%copy node 0 to top end
\pstVerb{ tx@Dict begin xcoeff 1 1 put ycoeff 1 1.5 put end }%
\pnode(! \psGetNodeCenter{\noderoot 1} xcoeff  1 \noderoot 1.x 1.5 mul put ycoeff 1  \noderoot 1.y 1.5 mul put 1 1){temp@}% 1.5=6*0.25
\multido{\iA=2+1}{\the\top@mone}{\pnode(! ycoeff \iA\space xcoeff  \iA\space \psGetNodeCenter{#1\iA} #1\iA.x 6 mul put #1\iA.y 6 mul put 1 1){temp@}}% move x, y coords times 6 to arrays
\multido{\iA=2+1,\iB=1+1}{\the\top@mtwo}{\pstVerb{ tx@Dict begin %
 ccoeff \iA\space ccoeff \iA\space get ccoeff \iB\space get sub mcoeff \iA\space get mul put %
 xcoeff \iA\space xcoeff \iA\space get xcoeff \iB\space get sub mcoeff \iA\space get mul put %
 ycoeff \iA\space ycoeff \iA\space get ycoeff \iB\space get sub mcoeff \iA\space get mul put
 bcoeff \iA\space bcoeff \iA\space get bcoeff \iB\space get mcoeff \iB\space get mul sub put
 bcoeff \the\top@ndx\space bcoeff \the\top@ndx\space get bcoeff \iB\space get ccoeff \iB\space get mul sub put
 xcoeff \the\top@ndx\space xcoeff \the\top@ndx\space get xcoeff \iB\space get bcoeff \iB\space get mul sub put
 ycoeff \the\top@ndx\space ycoeff \the\top@ndx\space get ycoeff \iB\space get bcoeff \iB\space get mul sub put end %
}% end \pstVerb
}% end \multido
\pstVerb{ tx@Dict begin %
 bcoeff \the\top@ndx\space bcoeff \the\top@ndx\space get bcoeff \the\top@mone\space get 
 ccoeff \the\top@mone\space get mul sub put %b(n)=b(n)-b(n-1)*c(n-1)
 xcoeff \the\top@ndx\space xcoeff \the\top@ndx\space get xcoeff \the\top@mone\space get  bcoeff \the\top@mone\space get mul sub bcoeff \the\top@ndx\space get div put %x(n)=(x(n)-x(n-1)*b(n-1))/b(n)
 ycoeff \the\top@ndx\space ycoeff \the\top@ndx\space get ycoeff \the\top@mone\space get bcoeff \the\top@mone\space get mul sub bcoeff \the\top@ndx\space get div put %y(n)=(y(n)-y(n-1)*b(n-1))/b(n)
 xcoeff \the\top@mone\space xcoeff \the\top@mone\space get xcoeff \the\top@ndx\space get ccoeff \the\top@mone\space get mul sub put %x(n-1)=x(n-1)-c(n-1)*x(n)
 ycoeff \the\top@mone\space ycoeff \the\top@mone\space get ycoeff \the\top@ndx\space get ccoeff \the\top@mone\space get mul sub put end %y(n-1)=y(n-1)-c(n-1)*y(n)
}% end pstVerb
\mmultido{\iA=\the\top@mone+-1,\iB=\the\top@ndx+-1}{\the\top@mtwo}{%
\pstVerb{ tx@Dict begin xcoeff \iA\space xcoeff \iA\space get mcoeff \iA\space get xcoeff \iB\space get mul sub ccoeff \iA\space get xcoeff \the\top@ndx\space get mul sub put % x(k)=x(k)-m(k)*x(k+1)-c(k)*x(n)
 ycoeff \iA\space ycoeff \iA\space get mcoeff \iA\space get ycoeff \iB\space get mul sub ccoeff \iA\space get ycoeff \the\top@ndx\space get mul sub put end %
}}%y(k)=...
\multido{\iA=1+1}{\the\top@ndx}{\pnode(! xcoeff \iA\space get ycoeff \iA\space get )%
{\noderoot B\iA}}%
\pnode(\noderoot B\the\top@ndx){\noderoot B0}%
}\ignorespaces}%
%
\def\thickBspline{\def\pst@par{}\pst@object{thickBspline}}%
\def\thickBspline@i#1#2#3#4{{%
% #1=root | #2=nsegments | #3=thickness | #4=items to clip
  \pst@killglue%
  \use@par%
  %prepare array
  \pssetlength{\pst@dimp}{#3}\pst@dimtonum{\pst@dimp}{\thickness}% thickness in TeX pt
  \newcount\n@cnt%
  \newcount\np@cnt%
  \newcount\ary@max \ary@max=\psk@plotpoints \relax%
  \multiply\ary@max #2\np@cnt=\ary@max \multiply\ary@max 4\relax%
  \advance\ary@max 4\relax %
  \n@cnt=\ary@max \advance\n@cnt \m@ne%
  \multiply\np@cnt \tw@\advance\np@cnt \tw@% count of points in array
  \pstVerb{ gsave tx@Dict begin %
    /psxu \pst@number\psxunit\space def /psyu \pst@number\psyunit\space def %
    /uratio psyu psxu div dup mul def % so that perp to a, b is -b*uratio, a
    /nseg #2 def /nplotpt \psk@plotpoints\space def %
    /clipary #2 nplotpt mul 4 mul 4 add array def %
    /lindex 2 def /uindex \the\ary@max\space 1 sub def %
    /dt 1 \psk@plotpoints\space div def %
  end grestore }%
  \multido{\iA=0+1,\iB=1+1}{#2}{%
% pass control points of this segment to PostScript variables
\pstVerb{% 
	gsave tx@Dict begin %
	STV CP T \psGetNodeCenter{#1\iA}\space \psGetNodeCenter{#1BR\iA}\space \psGetNodeCenter{#1BL\iB}\space \psGetNodeCenter{#1\iB}\space %
	/bez@1.x #1\iA.x def /bez@1.y #1\iA.y def %initial point on segment
	/bez@2.x #1BR\iA.x bez@1.x sub def %adjusted control points (less init pt)
	/bez@2.y #1BR\iA.y bez@1.y sub def %
	/bez@3.x #1BL\iB.x bez@1.x sub def %
	/bez@3.y #1BL\iB.y bez@1.y sub def %
	/bez@4.x #1\iB.x bez@1.x sub def %
	/bez@4.y #1\iB.y bez@1.y sub def %
	/bez@1x bez@2.x 3 mul def % for position--w1=3z1
	/bez@1y bez@2.y 3 mul def %
	/bez@2x bez@3.x bez@2.x 2 mul sub 3 mul def %position w2=3(z2-2z1)
	/bez@2y bez@3.y bez@2.y 2 mul sub 3 mul def %
	/bez@2xT bez@2x 2 mul def % for tangent
	/bez@2yT bez@2y 2 mul def %
	/bez@3x bez@4.x bez@2.x bez@3.x sub 3 mul add def % w3=z3+3z1-3z2
	/bez@3y bez@4.y bez@2.y bez@3.y sub 3 mul add def %
	/bez@3xT bez@3x 3 mul def % tangent
	/bez@3yT bez@3y 3 mul def %
	/Func % position
 ( bez@3x t mul bez@2x add t mul bez@1x add t mul bez@1.x add psxu mul %
 bez@3y t mul bez@2y add t mul bez@1y add t mul bez@1.y add psyu mul ) cvx def %
%    /FuncT % tangent
%      (  bez@2xT t bez@3xT mul add t mul bez@1x add  %
%       bez@2yT t bez@3yT mul add t mul bez@1y add  ) cvx def
/FuncN % unit normal times delta
 ( bez@2yT t bez@3yT mul add t mul bez@1y add uratio mul neg 
 bez@2xT t bez@3xT mul add t mul bez@1x add  2 copy Pyth dup %
 3 1 roll div 3 1 roll div \thickness\space 2 div dup 3 1 roll mul 3 1 roll mul ) cvx  def %
 end grestore }%
\ifnum\iA=\z@ %
  \pnode(! /t 0 def FuncN /dy exch def /dx exch def Func 2 copy %
  /y0 ED /x0 ED %
  clipary 0 x0 dx add put clipary 1 y0 dy add put %
  clipary uindex y0 dy sub put clipary uindex 1 sub x0 dx sub put %
  /uindex uindex 2 sub def ){tmp@P}%
\fi%
\multido{\i=1+1}{\psk@plotpoints}{%
      %make array elements for this segment using #3
      \pnode(! /t dt \i\space mul def FuncN /dy ED /dx ED Func 2 copy /y0 ED /x0 ED %
      clipary lindex x0 dx add put clipary lindex 1 add y0 dy add put %
      clipary uindex y0 dy sub put clipary uindex 1 sub x0 dx sub put %
      /lindex lindex 2 add def /uindex uindex 2 sub def ){tmp@P}%
}% end multido \i
}% end multido \iA
% if periodic, we have to adjust 2 points so as not to leave a wedge in the 
% clipping region. We have to set Q_{nk} to Q_0 and R_{nk} to R_0
% R_{nk} is at position 2*nk, 2*nk+1 and Q_{nk} is at the next pair
%\ifbsp@per %
%  \pstVerb{%
%   tx@Dict begin /itemcnt #2 nplotpt mul 2 mul def %first item to change is index 2nk
%  clipary itemcnt clipary 0 get put /itemcnt itemcnt 1 add def %
%  clipary itemcnt clipary 1 get put /itemcnt itemcnt 1 add def %
%  clipary itemcnt clipary \the\n@cnt\space 1 sub get put /itemcnt itemcnt 1 add def %
%  clipary itemcnt clipary \the\n@cnt\space get put %
%  end }%
%\fi%
% make a clipping path in PS and pass commands #4
\ifbsp@per %periodic
 \psclip{\pscustom{%
  \code{ clipary aload pop moveto nseg nplotpt mul %
   { lineto } repeat closepath %
  moveto  nseg nplotpt mul { lineto } repeat closepath }}}%
  #4%
 \endpsclip%
\else % not periodic
 \psclip{\pscustom{%
  \code{ clipary aload pop moveto nseg nplotpt mul 2 mul %
  1 add { lineto } repeat closepath %
  \ifPst@reverseclip\space reversepath \fi}}}%
  #4%
 \endpsclip%
\fi%
}%
\ignorespaces}%
%
\def\bsp@setup#1#2#3#4{%
%#1=bsp root, #2=num #3=\top@ndx, #4=top@ndx-1, #4
  \pstVerb{tx@Dict begin gsave STV CP T /bsp@desc #2 array def }%
%  \def\num{#2}
  \multido{\iA=#4+-1,\iB=#3+-1}{#2}{%\show\num
    \pstVerb{% 
	  \psGetNodeCenter{#1S\iA}\space \psGetNodeCenter{#1R\iA}\space \psGetNodeCenter{#1L\iB}\space \psGetNodeCenter{#1S\iB}\space %
	  /bez@0x #1S\iA.x def /bez@0y #1S\iA.y def %initial point on segment
	  /bez@1x #1R\iA.x bez@0x sub 3 mul def % for position--w1=3z1
	  /bez@1y #1R\iA.y bez@0y sub 3 mul def
	  /bez@2x #1L\iB.x bez@0x sub  #1R\iA.x bez@0x sub 2 mul sub 3 mul def % w2=3(z2-2z1)
	  /bez@2y #1L\iB.y bez@0y sub  #1R\iA.y bez@0y sub 2 mul sub 3 mul def % w2=3(z2-2z1)
	  /bez@3x #1S\iB.x bez@1x add #1L\iB.x 3 mul sub bez@0x 2 mul add def % w3=z3+3z1-3z2
	  /bez@3y #1S\iB.y bez@1y add #1L\iB.y 3 mul sub bez@0y 2 mul add def % w3=z3+3z1-3z2
	  bsp@desc \iA\space mark bez@2y 2 mul bez@3y 3 mul bez@2x 2 mul bez@3x 3 mul bez@0y %
	  bez@1y bez@2y bez@3y bez@0x bez@1x bez@2x bez@3x ] put }% 
    }% end multido
    \pstVerb{ grestore end }% closed tx@Dict, which contains array named bsp@desc
}% bsp@setup
\def\bspcurvepointsE{\pst@object{bspcurvepointsE}}%
\def\bspcurvepointsE@i#1#2#3{\bspcurvepoints@ii{#1}{#2}{E}{#3}}%
\def\bspcurvepoints{\pst@object{bspcurvepoints}}%
\def\bspcurvepoints@i#1#2#3{\bspcurvepoints@ii{#1}{#2}{}{#3}}%
\def\bspcurvepoints@ii#1#2#3#4{{%optional [plotpoints=xx]
%  #1=bsp root,#2=maxindex,#3=E/{},#4=root name for new PS arrays,
  \pst@killglue%
  \newcount\bsp@numndx
  \bsp@numndx=#2 \relax%
  \ifx#3E\relax 
    \advance \bsp@numndx \m@ne \edef\top@ndx{\the\bsp@numndx}
    \advance \bsp@numndx \m@ne \edef\topm@ndx{\the\bsp@numndx}%
  \else%
    \edef\top@ndx{\the\bsp@numndx}%
    \pst@cntc=\bsp@numndx\advance\pst@cntc \m@ne%
    \edef\topm@ndx{\the\pst@cntc}%
  \fi%
  \pst@cntb=\bsp@numndx %\advance\pst@cntb \@ne\relax%
  % set up the bsp ps array
  \edef\cmd{\noexpand\bsp@setup{#1}{\the\bsp@numndx}{\top@ndx}{\topm@ndx}}\cmd%
  \use@par%
  \pst@cntc=\psk@plotpoints\relax%\psk@plotpoints=plotpoints-1
  \pst@cnta=\pst@cntc \multiply\pst@cnta by \pst@cntb%
  \edef\bsp@nsegs{\the\bsp@numndx}%\show\bsp@nsegs%
  \advance\pst@cnta by \@ne\relax%=(plotpoints-1)*nsegs+1
  \edef\@arraysize{\the\pst@cnta}%
  \pstVerb{ tx@Dict begin %
    /psxu \pst@number\psxunit\space def /psyu \pst@number\psyunit\space def %
	/unitratio \pst@number\psyunit \pst@number\psxunit div def %
	/unitratiosq unitratio dup mul def %
    /dt 1 \psk@plotpoints\space div def %
    /#4.X \@arraysize\space array def %
    /#4.Y \@arraysize\space array def %
    /#4Delta.X \@arraysize\space array def %
    /#4Delta.Y \@arraysize\space array def %
    /#4Normal.X \@arraysize\space array def %
    /#4Normal.Y \@arraysize\space array def %
    /theseg 0 def %\the\bsp@numndx\space 1 sub def %
    bsp@desc theseg get % the first segment
    /cnt 1 def % counter for arrays
    dup dup dup 8 get 4 1 roll 4 get 3 1 roll %
    9 get exch 5 get % initial x, y, x', y'
    unitratiosq mul neg #4Normal.X 0 3 -1 roll put #4Normal.Y 0 3 -1 roll put %
    2 copy #4.Y 0 3 -1 roll put #4.X 0 3 -1 roll put %
    /priory ED /priorx ED % arrays now initialized
    \bsp@nsegs\space {% repeat nsegs times
      bsp@desc theseg get aload pop /x3 ED /x2 ED /x1 ED /x0 ED %
      /y3 ED /y2 ED /y1 ED /y0 ED %
      /xT3 ED /xT2 ED /yT3 ED /yT2 ED %
      /Func ( x3 t mul x2 add t mul x1 add t mul x0 add %
      y3 t mul y2 add t mul y1 add t mul y0 add ) cvx def %
      /FuncN ( yT2 t yT3 mul add t mul y1 add  unitratiosq mul neg %
      xT2 t xT3 mul add t mul x1 add ) cvx def %
      /theseg theseg 1 add def %
      /j 1 def \psk@plotpoints\space {% repeat for each plotpt
        /t j dt mul def Func 2 copy 2 copy FuncN % x y x y x y xN yN
        #4Normal.Y cnt 3 -1 roll put %
        #4Normal.X cnt 3 -1 roll put %
        #4.Y cnt 3 -1 roll put %
        #4.X cnt 3 -1 roll put % x y x y on stack
        priory sub #4Delta.Y cnt 3 -1 roll put %
        priorx sub #4Delta.X cnt 3 -1 roll put % x y on stack
        /priory ED /priorx ED /j j 1 add def /cnt cnt 1 add def } repeat %
     } repeat %#4.Y == %
   end }% end pstVerb
  \pst@cnta=\@arraysize \relax\advance\pst@cnta \m@ne %
  \expandafter\xdef \csname #4pointcount\endcsname {\the\pst@cnta}%
  \typeout{Created points #40 .. #4\the\pst@cnta}%
}\ignorespaces}%
\def\bspFnNode#1#2#3#4{% 
% #1=root name for B-spline control points, #2=top index
% #3=x value, #4=name of node to place at (x,f(x))
% This works only for a bspline function graph, not a general bspline curve
\pnode(! bsp@desc length /n ED %
    /x #3 def %
    x bsp@desc 0 get 8 get dup /xt ED lt { /x xt def } if %
    x bsp@desc n 1 sub get 8 4 getinterval aload pop add add add %endpt of last segment
    dup /xt ED gt { /x xt def } if %
    /j 0 def %
    n {% repeat
      x bsp@desc j get 8 get dup /xt ED le { exit } if /j j 1 add def %
    } repeat %
    j 0 gt { /j j 1 sub def } if %j is index of bezier segment containing x
    bsp@desc j get dup dup dup %
    11 get exch 10 get 4 2 roll 9 get exch 8 get x sub cubic_roots %
    /t ED pop pop %
    t 1 gt { /t 1 def } if % zroot is the t value at which x=x(t)
    x bsp@desc j get 4 4 getinterval aload pop % leaves y0, y1, y2, y3 on stack 
    t mul add t mul add t mul add % x y now on stack    
    ){#4}%
}% end bspFnNode
%
\def\bspNode#1#2#3#4{%
%src maxindex t target
\pnode(! bsp@desc length /n ED %
    /t #3 def %
    t 0 lt { /t 0 def } if %
    t n gt { /t n def } if % clamp t to [0,n]
    /j t cvi def j n eq { /j n 1 sub def } if %
    /t t j sub def % t in [0,1]
    bsp@desc j get dup %
    8 4 getinterval aload pop t mul add t mul add t mul add % desc[[j]] x(t)
    exch 4 4 getinterval aload pop t mul add t mul add t mul add % x(t) y(t)
    ){#4}%
}% end bspNode
%
\def\bspcurvenodes#1#2{%
%#1= basename for points on curve, #2=basename for nodes
\bsp@cntA=\csname #1pointcount\endcsname \advance\bsp@cntA \@ne %
\multido{\iA=0+1}{\bsp@cntA}{%
\pnode(! #1.X \iA\space get #1.Y \iA\space get ){#2\iA}}%
}%
%
% start experimental variable thickness
\def\thickBEspline{\def\pst@par{}\pst@object{thickBEspline}}%
\def\thickBEspline@i#1#2#3#4#5#6{{%
% #1=root | #2=nsegments | #3=maxthickness | #4=minthickness | #5=axis angle | #6=items to clip
% An axis angle of -45 means that lines normal to -45 will be minthickness and lines normal to 45 will be maxthickness.
  \pst@killglue%
  \use@par%
  %prepare array
  \pssetlength{\pst@dimp}{#3}\setlength{\pst@dimp}{0.5\pst@dimp}%
  \pssetlength{\pst@dimn}{#4}\setlength{\pst@dimn}{0.5\pst@dimn}%
  \advance\pst@dimp \pst@dimn\setlength{\pst@dimp}{0.5\pst@dimp}%
  \pst@dimtonum{\pst@dimp}{\ctrthick}% center of thickness/2 in tex pt
  \advance\pst@dimp -\pst@dimn%  
  \pst@dimtonum{\pst@dimp}{\deltathick}% (max-min)/2 in tex pt
  \newcount\n@cnt%
  \newcount\np@cnt%
  \newcount\ary@max \ary@max=\psk@plotpoints \relax%
  \multiply\ary@max #2\np@cnt=\ary@max \multiply\ary@max 4\relax%
  \advance\ary@max 4\relax %
  \n@cnt=\ary@max \advance\n@cnt \m@ne%
  \multiply\np@cnt \tw@\advance\np@cnt \tw@% count of points in array
  \pstVerb{ gsave tx@Dict begin %
    /psxu \pst@number\psxunit\space def /psyu \pst@number\psyunit\space def %
    /uratio psyu psxu div dup mul def % so that perp to a, b is -b*uratio, a
    /nseg #2 def /nplotpt \psk@plotpoints\space def %
    /clipary #2 nplotpt mul 4 mul 4 add array def %
    /lindex 2 def /uindex \the\ary@max\space 1 sub def %
    /dt 1 \psk@plotpoints\space div def %
  end grestore }%
\multido{\iA=0+1,\iB=1+1}{#2}{%
% pass control points of this segment to PostScript variables
\pstVerb{% 
	gsave tx@Dict begin %
	STV CP T \psGetNodeCenter{#1\iA}\space \psGetNodeCenter{#1BR\iA}\space \psGetNodeCenter{#1BL\iB}\space \psGetNodeCenter{#1\iB}\space %
	/bez@1.x #1\iA.x def /bez@1.y #1\iA.y def %initial point on segment
	/bez@2.x #1BR\iA.x bez@1.x sub def %adjusted control points (less init pt)
	/bez@2.y #1BR\iA.y bez@1.y sub def %
	/bez@3.x #1BL\iB.x bez@1.x sub def %
	/bez@3.y #1BL\iB.y bez@1.y sub def %
	/bez@4.x #1\iB.x bez@1.x sub def %
	/bez@4.y #1\iB.y bez@1.y sub def %
	/bez@1x bez@2.x 3 mul def % for position--w1=3z1
	/bez@1y bez@2.y 3 mul def %
	/bez@2x bez@3.x bez@2.x 2 mul sub 3 mul def %position w2=3(z2-2z1)
	/bez@2y bez@3.y bez@2.y 2 mul sub 3 mul def %
	/bez@2xT bez@2x 2 mul def % for tangent
	/bez@2yT bez@2y 2 mul def %
	/bez@3x bez@4.x bez@2.x bez@3.x sub 3 mul add def % w3=z3+3z1-3z2
	/bez@3y bez@4.y bez@2.y bez@3.y sub 3 mul add def %
	/bez@3xT bez@3x 3 mul def % tangent
	/bez@3yT bez@3y 3 mul def %
	/Func % position
 ( bez@3x t mul bez@2x add t mul bez@1x add t mul bez@1.x add psxu mul %
 bez@3y t mul bez@2y add t mul bez@1y add t mul bez@1.y add psyu mul ) cvx def %
%    /FuncT % tangent
%      (  bez@2xT t bez@3xT mul add t mul bez@1x add  %
%       bez@2yT t bez@3yT mul add t mul bez@1y add  ) cvx def
/FuncN % unit normal times delta
 ( bez@2yT t bez@3yT mul add t mul bez@1y add uratio mul neg 
 bez@2xT t bez@3xT mul add t mul bez@1x add  2 copy Pyth dup %
 3 1 roll div 3 1 roll div 2 copy Atan #5 sub 2 mul cos \deltathick\space mul neg \ctrthick\space add dup 3 1 roll mul 3 1 roll mul ) cvx  def %
 end grestore }%
\ifnum\iA=\z@%
  \pnode(! /t 0 def FuncN /dy exch def /dx exch def Func 2 copy %
  /y0 ED /x0 ED %
  clipary 0 x0 dx add put clipary 1 y0 dy add put %
  clipary uindex y0 dy sub put clipary uindex 1 sub x0 dx sub put %
  /uindex uindex 2 sub def ){tmp@P}%
\fi%
\multido{\i=1+1}{\psk@plotpoints}{%
      %make array elements for this segment
      \pnode(! /t dt \i\space mul def FuncN /dy ED /dx ED Func 2 copy /y0 ED /x0 ED %
      clipary lindex x0 dx add put clipary lindex 1 add y0 dy add put %
      clipary uindex y0 dy sub put clipary uindex 1 sub x0 dx sub put %
      /lindex lindex 2 add def /uindex uindex 2 sub def ){tmp@P}%
}% end multido \i
}% end multido \iA
% if periodic, we have to adjust 2 points so as not to leave a wedge in the 
% clipping region. We have to set Q_{nk} to Q_0 and R_{nk} to R_0
% R_{nk} is at position 2*nk, 2*nk+1 and Q_{nk} is at the next pair
%\ifbsp@per %
%  \pstVerb{%
%   tx@Dict begin /itemcnt #2 nplotpt mul 2 mul def %first item to change is index 2nk
%  clipary itemcnt clipary 0 get put /itemcnt itemcnt 1 add def %
%  clipary itemcnt clipary 1 get put /itemcnt itemcnt 1 add def %
%  clipary itemcnt clipary \the\n@cnt\space 1 sub get put /itemcnt itemcnt 1 add def %
%  clipary itemcnt clipary \the\n@cnt\space get put %
%  end }%
%\fi%
% make a clipping path in PS and pass commands #6
\ifbsp@per %periodic
 \psclip{\pscustom{%
  \code{ clipary aload pop moveto nseg nplotpt mul %
   { lineto } repeat closepath %
  moveto  nseg nplotpt mul { lineto } repeat closepath }}}%
  #6%
 \endpsclip%
\else % not periodic
 \psclip{\pscustom{%
  \code{ clipary aload pop moveto nseg nplotpt mul 2 mul %
  1 add { lineto } repeat closepath %
  \ifPst@reverseclip\space reversepath \fi}}}%
  #6%
 \endpsclip%
\fi%
}%
\ignorespaces}%
%
\def\thickBdraw#1#2#3#4#5{\edef\n@decnt{\csname #1nodecount\endcsname}%
\psBsplineInterp{#1}{\n@decnt}%
\psBsplineNodes[linestyle=none,showpoints=false]{#1B}{\n@decnt}%
\thickBEspline{#1}{\n@decnt}{#2}{#3}{#4}{#5}%
}%

% Elliptic pen as possible in metafont/metapost
\def\thickBsplinePen{\def\pst@par{}\pst@object{thickBsplinePen}}%
\def\thickBsplinePen@i#1#2#3#4#5#6{%
% #1=root | #2=nsegments | #3=semimajor | #4=semiminor | #5=inclination | #6=items to clip
% An inclination of -45 means that lines pointing at angles -45 and 135 thinnest and lines pointing 45 and 225 will be thickest, assuming #3>#4.
  \pst@killglue%
  \psBsplineInterp{#1}{#2}%
  \psBsplineNodes[linestyle=none]{#1B}{#2}%
  \thickBsplinePen@ii{#1}{#2}{#3}{#4}{#5}{#6}%
}%
\def\thickBsplinePen@ii#1#2#3#4#5#6{{%
  \pst@killglue%
  \use@par%
  \ifdim\psxunit=\psyunit \else \errmessage{Unequal units--must bail out from thickBsplinePen}\fi%
  %prepare array
  \pssetlength{\pst@dimp}{#3}\pst@dimtonum{\pst@dimp}{\bez@a}%
  \pssetlength{\pst@dimn}{#4}\pst@dimtonum{\pst@dimn}{\bez@b}%
  \newcount\n@cnt%
  \newcount\np@cnt%
  \newcount\ary@max \ary@max=\psk@plotpoints \relax%
  \multiply\ary@max #2\np@cnt=\ary@max \multiply\ary@max 4\relax%
  \advance\ary@max 4\relax %
  \n@cnt=\ary@max \advance\n@cnt \m@ne%
  \multiply\np@cnt \tw@\advance\np@cnt \tw@% count of points in array
  \pstVerb{ gsave tx@Dict begin %
    /psxu \pst@number\psxunit\space def /psyu \pst@number\psyunit\space def %
    /uratio psyu psxu div dup mul def % so that perp to a, b is -b*uratio, a
    /nseg #2 def /nplotpt \psk@plotpoints\space def %
    /clipary #2 nplotpt mul 4 mul 4 add array def %
    /lindex 2 def /uindex \the\ary@max\space 1 sub def %
    /dt 1 \psk@plotpoints\space div def %
  end grestore }%
\multido{\iA=0+1,\iB=1+1}{#2}{%
% pass control points of this segment to PostScript variables
\pstVerb{% 
	gsave tx@Dict begin %
	STV CP T \psGetNodeCenter{#1\iA}\space \psGetNodeCenter{#1BR\iA}\space \psGetNodeCenter{#1BL\iB}\space \psGetNodeCenter{#1\iB}\space %
	/bez@1.x #1\iA.x def /bez@1.y #1\iA.y def %initial point on segment
	/bez@2.x #1BR\iA.x bez@1.x sub def %adjusted control points (less init pt)
	/bez@2.y #1BR\iA.y bez@1.y sub def %
	/bez@3.x #1BL\iB.x bez@1.x sub def %
	/bez@3.y #1BL\iB.y bez@1.y sub def %
	/bez@4.x #1\iB.x bez@1.x sub def %
	/bez@4.y #1\iB.y bez@1.y sub def %
	/bez@1x bez@2.x 3 mul def % for position--w1=3z1
	/bez@1y bez@2.y 3 mul def %
	/bez@2x bez@3.x bez@2.x 2 mul sub 3 mul def %position w2=3(z2-2z1)
	/bez@2y bez@3.y bez@2.y 2 mul sub 3 mul def %
	/bez@2xT bez@2x 2 mul def % for tangent
	/bez@2yT bez@2y 2 mul def %
	/bez@3x bez@4.x bez@2.x bez@3.x sub 3 mul add def % w3=z3+3z1-3z2
	/bez@3y bez@4.y bez@2.y bez@3.y sub 3 mul add def %
	/bez@3xT bez@3x 3 mul def % tangent
	/bez@3yT bez@3y 3 mul def %
	/Func % position
 ( bez@3x t mul bez@2x add t mul bez@1x add t mul bez@1.x add psxu mul %
 bez@3y t mul bez@2y add t mul bez@1y add t mul bez@1.y add psyu mul ) cvx def %
%    /FuncT % tangent
%      (  bez@2xT t bez@3xT mul add t mul bez@1x add  %
%       bez@2yT t bez@3yT mul add t mul bez@1y add  ) cvx def
/FuncN % leave dx, dy on stack
 ( bez@2yT t bez@3yT mul add t mul bez@1y add dup /ty ED uratio mul neg 
 bez@2xT t bez@3xT mul add t mul bez@1x add dup /tx ED %
 2 copy Pyth dup  3 1 roll div 3 1 roll div /nx ED /ny ED % unit normal to curve
 tx ty 2 copy Pyth dup 3 1 roll div 3 1 roll div /tx ED /ty ED % unit tangent
 #5 ty tx Atan sub dup cos /bez@c ED sin /bez@s ED %
 \bez@a\space bez@s mul \bez@b\space bez@c mul Pyth % sqrt(aa*ss+bb*cc)
 dup .001 lt { pop \bez@a\space tx mul \bez@a\space ty mul }{/yval ED %
 bez@c bez@s mul \bez@a\space dup mul \bez@b\space dup mul sub mul yval div %
 /xval ED %
 xval tx mul yval nx mul add xval ty mul yval ny mul add } %
 ifelse ) cvx  def %
 end grestore }%
\ifnum\iA=\z@%
  \pnode(! /t 0 def FuncN /dy exch def /dx exch def Func 2 copy %
  /y0 ED /x0 ED %
  clipary 0 x0 dx add put clipary 1 y0 dy add put %
  clipary uindex y0 dy sub put clipary uindex 1 sub x0 dx sub put %
  /uindex uindex 2 sub def ){tmp@P}%
\fi%
\multido{\i=1+1}{\psk@plotpoints}{%
      %make array elements for this segment
      \pnode(! /t dt \i\space mul def FuncN /dy ED /dx ED Func 2 copy /y0 ED /x0 ED %
      clipary lindex x0 dx add put clipary lindex 1 add y0 dy add put %
      clipary uindex y0 dy sub put clipary uindex 1 sub x0 dx sub put %
      /lindex lindex 2 add def /uindex uindex 2 sub def ){tmp@P}%
}% end multido \i
}% end multido \iA
% if periodic, we have to adjust 2 points so as not to leave a wedge in the 
% clipping region. We have to set Q_{nk} to Q_0 and R_{nk} to R_0
% R_{nk} is at position 2*nk, 2*nk+1 and Q_{nk} is at the next pair
%\ifbsp@per %
%  \pstVerb{%
%   tx@Dict begin /itemcnt #2 nplotpt mul 2 mul def %first item to change is index 2nk
%  clipary itemcnt clipary 0 get put /itemcnt itemcnt 1 add def %
%  clipary itemcnt clipary 1 get put /itemcnt itemcnt 1 add def %
%  clipary itemcnt clipary \the\n@cnt\space 1 sub get put /itemcnt itemcnt 1 add def %
%  clipary itemcnt clipary \the\n@cnt\space get put %
%  end }%
%\fi%
% make a clipping path in PS and pass commands #6
\ifbsp@per %periodic
 \psclip{\pscustom{%
  \code{ clipary aload pop moveto nseg nplotpt mul %
   { lineto } repeat closepath %
  moveto  nseg nplotpt mul { lineto } repeat closepath }}}%
  #6%
 \endpsclip%
\else % not periodic
 \psclip{\pscustom{%
  \code{ clipary aload pop moveto nseg nplotpt mul 2 mul %
  1 add { lineto } repeat closepath %
  \ifPst@reverseclip\space reversepath \fi}}}%
  #6%
 \endpsclip%
\fi%
}%
\ignorespaces}%
%
% Elliptic pen as possible in metafont/metapost, ignoring end segs
\def\thickBsplinePenE{\def\pst@par{}\pst@object{thickBsplinePenE}}%
\def\thickBsplinePenE@i#1#2#3#4#5#6{{%
% #1=root | #2=nsegments | #3=semimajor | #4=semiminor | #5=inclination | #6=items to clip
% An inclination of -45 means that lines pointing at angles -45 and 135 thinnest and lines pointing 45 and 225 will be thickest, assuming #3>#4.
  \pst@killglue%
  \edef\n@decnt{\csname #1nodecount\endcsname}
  \psBsplineInterp{#1}{\n@decnt}%
  \psBsplineNodesE[linestyle=none]{#1B}{\n@decnt}%
  \count0=\n@decnt\advance\count0\m@ne\edef\n@decntm{\the\count0}%
  \advance\count0\m@ne\edef\newn@decnt{\the\count0}%
  \multido{\iA=0+1,\iB=1+1}{\n@decntm}{\pnode(#1\iB){@#1\iA}%
  \pnode(#1B\iB){@#1B\iA}\pnode(#1BR\iB){@#1BR\iA}\pnode(#1BL\iB){@#1BL\iA}}%
  \use@par%
  \thickBsplinePen@ii{@#1}{\newn@decnt}{#3}{#4}{#5}{#6}%
}\ignorespaces}%
%
\catcode`\@=\TheAtCode\relax
\endinput