%%
%% This is file `bpfats1.tex',
%% generated with the docstrip utility.
%%
%% The original source files were:
%%
%% stepe.dtx  (with options: `bpfats1')
%% 
%%     This work has been partially funded by the US government
%%  and is not subject to copyright.
%% 
%%     This program is provided under the terms of the
%%  LaTeX Project Public License distributed from CTAN
%%  archives in directory macros/latex/base/lppl.txt.
%% 
%%  Author: Peter Wilson (CUA and NIST)
%%          now at: peter.r.wilson@boeing.com
%% 
\ProvidesFile{bpfats1.tex}[2001/07/16 ATS end intro boilerplate]
\typeout{bpfats1.tex [2001/07/16 ATS end intro boilerplate]}

The purpose of an abstract test suite is to provide a basis for
evaluating whether a particular implementation of an application
protocol actually conforms to the requirements of that application
protocol. A standard abstract test suite helps ensure that
evaluations of conformance are conducted in a consistent manner
by different test laboratories.

This part of ISO~10303 specifies the abstract test suite for
ISO 10303-\theAPpartno, application protocol \theAPtitle.
The abstract test cases presented here are the basis for
conformance testing of implentations of ISO 10303-\theAPpartno.

    This abstract test suite is made up of two major parts:
\begin{itemize}
\item the test purposes, the specific items to be covered by
      conformance testing;
\item the set of abstract test cases that meet those test purposes.
\end{itemize}

    The test purposes are statements of the application protocol
requirements that are to be addressed by the abstract test cases.
Test purposes are derived primarily from the application protocol's
information requirements and AIM,
as well as from other sources such as standards
referenced by the application protocol and other requirements
stated in the application protocol conformance requirements clause.

    The abstract test cases address the test purpose by:
\begin{itemize}
\item specifying the requirements for input data to be used when
      testing an implementation of the application protocol;
\item specifying the verdict criteria to be used when evaluating
      whether the implementation successfully converted the input
      data to a different form.
\end{itemize}

    The abstract test cases set the requirements for the
executable test cases that are required to actually conduct
a conformance test. Executable test cases contain the scripts,
detailed values, and other explicit information required to
conduct a conformance test on a specific implementation of
the application protocol.

    At the time of publication of this document, conformance
testing requirements had been established for implementations
of application protocols in combination with ISO 10303-21 and
ISO 10303-22. This part of ISO 10303 only specifies
test purposes and abstract test cases for a subset of such
implementations.

    For ISO 10303-21, two kinds of implementations, preprocessors and
postprocessors, must be tested. Both of these are addressed in this
abstract test suite.

    For ISO 10303-22, a class of applications will possess the capability
to upload and download AP-compliant SDAI-models or schema instances
to and from applications that implement the SDAI. By providing test case
data that correspond with SDAI-models, this abstract test suite addresses
such applications in a single-schema scenario.

\endinput
%%
%% End of file `bpfats1.tex'.
